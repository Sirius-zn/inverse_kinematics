\documentclass{article}

\usepackage[margin=1in]{geometry}
\usepackage{setspace}

\title{Final Project - Status Update 1}
\author{Michael Forney \\ SID: 21392560}

\begin{document}
    \maketitle
    \doublespacing

    I have just about finished implementing my algorithm for solving the
    inverse kinematics system. I have chosen to use a variation on the cyclic
    coordinate descent algorithm to iteratively improve the joint angles and
    displacements, supporting constraints on the rotation of a revolute joint,
    and the displacement of a prismatic joint.

    My solution also allows for arbitrary branching from the end of a rod. It
    works by finding the critical points of the equation representing error
    (least squares), and then uses the point with the lowest error. I also
    include the minimum and maximum constraint to account for the case where
    the solution does not lie in the constraints of the joint.

    I have a few bugs left to fix in my implementation of cyclic coordinate
    descent, but it seems to be pretty close to working. I still have to work
    on the user interface to allow users to interact with the structure to
    specify joint angles and target joint positions. This is looking more
    complex than I originally anticipated, but I think it will still be
    manageable. I also need to work on loading and saving pose configurations,
    and simple timeline editing for animations. Cairo, the vector graphics
    library I'm using, allows for easily rendering to png images, so video
    output (through images) will be trivial.

\end{document}

